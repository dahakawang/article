\chapter{绪论}
\section{背景和意义}
随着网络技术的持续发展,基于Web的网络应用开始崛起。凭借其简洁的开发技术,高效的配套工具以及较高的易访问性,Web应用在很多领域开始逐步取代传统桌面型应用,成为许多开发者进行技术选型的不二选择。然而,虽然相比起传统桌面应用,Web应用拥有诸多优势,它的劣势也十分明显。其中之一便是缺少像PC桌面应用一般对电脑的高度可控性:由于Web应用构建于网络之上,使得Web应用的通信机制囿于HTTP协议本身的内容,很难实现从服务器端到客户端的消息推送。

而对一款现代的Web应用来讲,能够实时地从服务器端获取数据是十分必要的\supercite{why-live}。特别是对于微博、金融软件、炒股软件、网站邮件箱等在线系统来说,使用一种异步消息推送机制是不可或缺的。

就目前而言,为了实现服务器异步的消息推送,主流的方法是使用Ajax技术\supercite{ajax-book}。Ajax即“异步JavaScript和XML技术(Asynchronous JavaScript and XML)”,是Jesse James Garrett于2005年首次提出的一项技术,该技术可用于实现即时消息推送\supercite{ajax-intro}。Ajax需要不断的从服务器端获取数据,这种类似轮询的方式在Web环境下的效率是十分低下的。特别是由于每次请求均会伴随新的HTTP请求产生,因此每次请求都会经过“三次握手”等繁琐的初始化过程,这既带来了额外网络流量开销,也影响了程序执行性能和效率\supercite{ajax-apple}\supercite{ajax-sse}。

针对此种情况,W3C提出了一种称之为SSE的技术来解决\supercite{sse-w3c}。SSE是由W3C组织制定的一套从服务器到浏览器的通信标准,是新的HTML5标准的一部分,它既是一套协议和标准,亦代表了一大类的实现技术。由于SSE仅会在初始化时建立HTTP连接,因此避免了类似于Ajax技术的低效率性。

实现SSE标准对Web服务器来说是十分有价值的。就目前而言,绝大多数的浏览器均已开始支持HTML5标准,这意味着SSE已经被许多浏览器支持,由此带来的是SSE技术的原生实现。相比起Ajax需要使用JavaScript实现处理逻辑,由浏览器原生实现的SSE标准则是具有更高的性能和容错性。对前端开发者来说,由于浏览器内建支持,这亦是一种减少其工作量的技术选择。

由于SSE如此重要,Ruby on Rails框架亦考虑将其纳入下一个版本的实现,这便是Rails消息总线技术的成因。Rails消息总线技术是一套通用的从Web服务器到浏览器的数据推送解决方案,它的主要目标是实现Rails服务器的SSE标准,但同时也能为开发者工具的建设提供支持。利用Rails消息总线技术,本文实现了Rails开发模式下帮助开发者调试的两套工具:Notification Based Code Auto-Reload以及Backend Instrumentaion API。

\section{论文主要工作}
论文在学习和总结了传统服务器消息推送机制的基础上,结合开源框架Ruby on Rails的具体需求,设计了一套合适于Rails开发者的数据推送系统。同时,为验证该系统的正确性并将其应用于实践之中,本文还开发了一款在线应用的示例程序以及两款开发者工具。这些工作可以被详细描述为一下几个方面:
\begin{enumerate}
\item 了解和调研了目前Web开发领域中主流的服务器数据推送技术,并对各种方法的优劣做了详细的总结和对比。
\item 修改了Ruby on Rails依赖库Rack的代码,以使得本技术能够在Rails上顺利实现并适配于Rack。并于GitHub上将修改提交到了Rack官方代码库中。
\item 在Ruby on Rail框架内实现了Rails消息总线技术,并将修改合并到了Rails官方开发开发分支中
\item 基于本技术,为Ruby on Rails开发了两套调试工具:Notification Based Code Auto-Reload以及Backend Instrumentaion API。
\item 基于本技术,开发了一款示例网页应用,以便对其进行验证和展示
\end{enumerate}

\section{论文组织与结构}
本文的组织结构如下所示:

第一部分,绪论。简单地介绍了数据推送机制的现实意义,分析了目前主流方法的优劣,阐述了本技术开发的背景和意义,列举了本文所做的相关工作。

第二部分:基础技术研究。详细地介绍了本技术在实现过程当做所使用的各种技术,提供了充足的技术背景资料。主要介绍了Ruby语言、Ruby开发工具及环境、元编程、Ruby on Rails开源框架、Rack劫持技术、SSE标准等技术的内涵,并适当分析了其和本技术密切相关的实现原理。

第三部分:系统总体设计。该部分首先详尽分析了本技术需要支持应用场景,然后在此需求的基础上详细描述了本技术总体上的实现方法。为了更为清晰的阐明本技术的宏观实现,该部分还详尽地分析了本技术的动态逻辑结构、静态逻辑结构以及部署时的物理结构。

第四部分:该部分详细地分析了本技术各个组件和子系统是如何设计及协同工作的。主要介绍了FileSystemChecker组件、MessageServer子系统、ActionController::Live组件、以及MessageBus Gem的设计。

第五部分:系统测试。由于本技术属于Rails基础技术,为了测试其正确性以及演示其功能性,本文基于Rails消息总线技术,开发了Auto-Reload、Backend Instrumentation以及一个网页示例程序,并以此为据进行了测试。给出了测试结果,并对结果和预期给出了分析。

第六部分:总结。首先简要回顾了一下本次工作所涉及技术的特点,归纳了本技术的优势和缺点,提出了本技术未来改进思路。然后总结了本次撰写论文工作中遇到的困难,总结了其间的不足和应该加以改进的地方。