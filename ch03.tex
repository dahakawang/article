\chapter{系统总体设计}
\section{目标分析}
本小节主要分析了Rails消息总线项目的具体需求。本项目由三个子项目构成,Rails消息总线子项目用于提供从Rails服务器到浏览器的基础服务技术;Auto-Reload技术主要用于开发模式时自动检测源代码更改事件并通知浏览器进行刷新动作;而Backend Instrumentation技术则是通过在后台采集性能数据,并将其反馈到前端的技术。

\subsection{Rails消息总线需求分析}
Rails消息总线提供了一套机制,使得服务器能够实时地推送任意数据到浏览器客户端。本文从两个层面上实现了Rails消息总线技术,这两个层面抽象程度、易用性以及灵活性各不相同,适应于不同的应用场景。针对这两个层面,Rails消息总线技术提供了两套接口(模式)。 这两套接口中,抽象程度较低、更为灵活的是Raw Streaming Mode,使用这套接口,服务器可以发送任意格式的数据,HTTP报文的报头和内容完全由服务器定制而不受Rails框架限制。另一套接口是SSE Mode,该套接口抽象程度较高,使用更为简单。SSE Mode即Server Sent Events技术的Rails实现版本,Rails消息总线技术实现了较为完整的SSE技术规范,对于规范中大部分内容(特别是开发环境中最常使用的特性)都予以了支持。

对于Raw Streaming模式,Rails消息总线技术允许程序获取服务器和浏览器连接的原始套接字对象。并且允许开发者直接向该套接字读写数据,并且写入的数据将被即刻传输至浏览器,从而实现了实时数据推送。为了使得服务器能够推送数据,首先需要浏览器主动连接到Rails服务器上以获取数据。该过程需要通过浏览器向Rails服务器发送一个传统的HTTP连接请求来完成,因此在服务器端,开发者必须首先定义一个入口Controller和相应的action。当入口Controller和action定义好,并且在Rails项目的route.rb文件中注册完成之后,浏览器可以通过地址“http://server\_addr/controller/action”连接上Rails服务器。此时由于该路径已在route.rb中注册成功,Rails会将该请求路由到相应的Controller及其action上的代码处。此时,Rails开发者可以通过在相应的action中获取response对象,通过向数据流response.stream写入数据,可以实现超浏览器发送实时数据。代码\ref{raw-streaming-demo}展示了如何使用该接口:

\begin{lstlisting}[caption={Raw Streaming Mode使用示例}, label=raw-streaming-demo]
class MyController < ActionController::Base
  include ActionController::Live

  def stream
    response.headers['Content-Type'] = 'text/event-stream'
    100.times {
      response.stream.write "hello world\n"
      sleep 1
    }
    response.stream.close
  end
end
\end{lstlisting}

代码\ref{raw-streaming-demo}首先定义了一个Controller和action用于接收浏览器的请求。此时浏览器可以通过地址"http://server\_addr/my/stream"地址来访问服务器实时数据推送服务。当浏览器请求被Rails服务器接收后,ActionController\#Stream则会被调用,以响应浏览器得请求。

为了使用Rails实时消息推送机制,代码\ref{raw-streaming-demo}通过include ActionController::Live语句将ActionController::Live模块激活,从而启用了消息推送机制。这之后,开发这可以通过response.stream来向客户端发送任意数据。当然,开发者亦能够通过response.headers来设置响应HTTP报文得报头内容,从而更精细化地控制消息。这里值得指出的是,开发者只能够在response状态变成提交之前设置HTTP报头,否则消息总线系统将抛出一个异常。response状态编程提交的时机,便是开发者第一次调用response.stream.write或者response.stream.close的时候。如果开发者调用这两个接口任意一个,则response状态变为已提交,此时则将不能够设置HTTP报头内容。

最后,如果客户端没有显式地断开连接,服务器在停止服务之前必须显式地调用response.stream.close接口以关闭数据流。否则将会照成和客户端之间的连接不能被中断,系统资源难以回收,从而影响到服务器系统执行效率。

Rails消息总线技术提供的另一套接口则是基于标准的SSE规范的。该接口的实现,一则是为了提供一个抽象程度较高、十分易用的接口给Rails开发者,二则是为了是Rails能够支持HTML5新规范,特别是对于支持HTML5规范中SSE的浏览器(HTML5兼容浏览器)提供支持。使用该套接口,Rails会自动生成兼容的SSE消息格式,同浏览器交换元数据,并且实现断线重连等机制。但是对上,本技术向Rails开发者隐藏了此类复杂机制,对于其来说,Rails消息总线机制提供了一个稳定的、安全的、健壮的通信机制,使得其能够实时的向浏览器发送SSE兼容消息。

SSE接口同样提供了两种模式,一种是Controller Level SSE,另一种则是Client Level SSE。对于前一种模式来说,Rails不区分连接到自己的不同客户端。当开发者发送SSE消息至客户端后,所有当前连接的客户端均会收到该条消息。使用Controller Level SSE,开发者不需要遍历客户端列表,并逐一发送SSE消息,这对于某些全局通用消息的发送场合能够大大提高效率。但是,由于Controller Level SSE对客户端不做区分,这使得系统无法针对不同客户端推送不同消息。而针对不同客户端推送不同消息在某些应用场景下是十分关键和重要的(例如微博消息推送、论坛和网站的即时消息等)。针对这类应用场景,Rails消息总线技术提供了Client Level SSE予以解决。通过使用Client Level SSE,开发者将在不同的线程中响应浏览器请求并发送即时消息。并且,开发者同时能够获取当前回话相关的变量,例如cookies、HTTP请求报头、HTTP请求变量等等和具体客户端相关联的数据。据此,开发者能够实现针对不同的客户端,推送不同的消息。

对于Controller Level SSE,可以通过下面的代码来使用之:

\begin{lstlisting}[caption={Controller Level SSE示例}, label=cls-demo]
# SSE entry controller
class MyController < ActionController::Base
  include ActionController::Live
  include ActionController::ServerSentEvents
  extend ActionController::ServerSentEvents::ClassMethods
end

# when we want to send an event
@sse = ServerSentEvent.new "this is data"
MyController.send_sse @sse

# send an event through another interface
MyController.send_sse_hash :data => "this is also data"
\end{lstlisting}

在代码\ref{cls-demo}中,为了定义一个Controller Level SSE的入口Controller和action,代码分别include了ActionController::Live和ActionController::ServerSentEvents两个模块,并且extend了ActionController::ServerSentEvents::ClassMethods模块。

通过上述三行代码,使得MyController成为了Controller Level SSE的入口。这时候一个叫做sse\_source的action将被自动定义,开发者需要将该action注册到route.rb文件中,从而使得Rails路由系统能够找到是sse\_source action。这时候,浏览器通过访问“http://server\_addr/my/sse\_source",则可以接收相应的SSE消息。

在代码\ref{cls-demo}同时展示了如何发送一条消息至客户端。可以看到,使用MyController\#send\_sse接口,需要首先生成一个ServerSentEvent对象,并将其作为参数发送;而若使用MyController\#send\_sse\_hash接口,则可以直接以一个哈希表为参数,此时哈希表中:data键所对应的数据则是待发送数据。值得指出的是,MyController\#send\_sse以及MyController\#send\_sse\_hash可以在任何线程中执行。其执行结果是,所有连接到“http://server\_addr/my/sse\_source”的浏览器都会接受到该SSE全局消息。

Rails消息总线为SSE提供的另一个接口便是Client Level SSE。代码\ref{cls2-demo}展示了如何使用该技术:

\begin{lstlisting}[caption={Client Level SSE示例}, label=cls2-demo]
class MySSE < ActionController::Base
  include ActionController::Live
  include ActionController::ServerSentEvents

  def event
    start_serve do |sse_client|
      # we can access some session variables here
      sse_client.send_sse sse
      sse_client.send_sse_hash :data => "david"
    end
  end
end
\end{lstlisting}

代码\ref{cls2-demo}首先定义了一个Controller及一个action,并手动在route.rb文件中注册相应的路由策略。这样,客户端浏览器便可以通过地址“http://server\_addr/my/event”连接到该数据源服务器上。在event方法内,只有一个start\_serve调用,该函数的执行将使得服务器一直持有对客户端浏览器之间的原始套接字,并且通过调用关联的回调方法向原始套接字中写入数据。可以看到,在调用start\_serve的同时,代码\ref{cls2-demo}传入了一个Block对象。该对象作为一个回调函数,用于处理实际向客户端原始套接字写入数据的任务。该回调函数拥有一个参数,该参数是一个sse\_client对象,借以该对象的send\_sse\_hash和send\_sse两个接口,开发者可以直接向客户端发送SSE消息。

值得指出的是,Client Level SSE和Controller Level SSE在消息发送接口上十分接近。前者通过回调函数的参数获取一个sse\_client对象,并通过向该对象的send\_sse\_hash和send\_sse两个接口从而实现向客户端浏览器发送SSE消息的功能;而后者则是通过在任意时刻通过Controller的两个类方法send\_sse\_hash和send\_sse,来实现向全部连接到该数据源的客户端浏览器发送SSE消息的功能。另外,可以看到,Client Level SSE是不需要extend模块ActionController::ServerSentEvents::ClassMethods的。这是因为Client Level SSE模式需要开发者自行定义和编写数据源入口action的代码。

\subsection{Auto-Reload需求分析}
Auto-Reload是针对于Rails开发者的辅助技术,它实现了自动检测Rails后台源码更改并通知浏览器即时刷新的功能。在现代的网站开发过程中,一个基本网站的构建依赖于众多技术的支持,这产生了对数量庞大的源代码文件的依赖。这些代码文件既包括网站后台逻辑的代码,也包括网页前台的HTML定义文件,层叠样式表CSS文件,前端逻辑的JavaScript文件等等。对这些文件当中的任何一个部分进行更改对整个网站来说,都是牵一发而动全身的。

传统上,由于对这些源文件更改的效果,需要网站开发者重新编译并且部署网站,方能够看到改动后的效果。这需要开发者投入大量的精力和时间,对于正在专注于代码逻辑编写的开发者来说,这一系列繁琐的过程势必使得其开发效率大打折扣。在Rails中,为了极大的提高Rails开发者的生产效率,本文引入了一项新的称之为Auto-Reload的技术。该技术充分利用了Ruby这门动态脚本语言的特性,结合Rails框架自身的灵活性,实现了后台代码一旦改动,前端调试页面立刻刷新而不需要开发者手动干预这么一个功能。

由于Ruby是一门非编译型的动态脚本语言,这使得对后台代码的更改能够立刻体现到程序逻辑之上而不需要重新经历一遍传统开发语言所必须经历的编译、链接、部署等过程。这为Auto-Reload的实现客观上创造了可能性。但是,为了引入改动代码所带来的变化,我们依旧需要浏览器重新向后台服务器请求新的页面。于是,Auto-Reload便承担了这项工作。Auto-Reload自动监测Rails工程文件夹下所有的文件系统消息,并筛选处一切对可能影响网站逻辑、外观的文件(.rb文件,.css文件,.js文件,.html文件等等)的变动消息。一旦发觉此种文件被改动了,则立刻通过Rails消息总线通知客户端浏览器文件的变动,并告知其立刻刷新。与此同时,在客户端浏览器中,本技术通过一个JavaScript脚本向后台Rails服务器订阅一个关于文件变动消息的SSE消息源。一旦JavaScript脚本接受到文件变动消息,则指导浏览器进行网页刷新动作,重新发送HTTP请求,载入新的网页页面、并重新渲染改动后网页。

Auto-Reload对外实现了良好的接口,只需要开发者在Rails工程的Gemfile里指出想要使用auto-reload即可。下面的代码演示了如何使用Auto-Relaod技术

\begin{lstlisting}[caption={启用Auto-Realod技术}, label=auto-reload-demo]
source 'https://rubygems.org'

gem 'rack', path: '/Users/david/work_place/projects/rack'
gem 'rails',     path: '/Users/david/work_place/projects/rails'
gem 'arel',      github: 'rails/arel'
gem 'activerecord-deprecated_finders', github: 'rails/activerecord-deprecated_finders'
gem 'debugger'
gem 'sqlite3'
gem 'puma'

# To enable Auto-Reload feature
gem 'autorelaod', path: '/Users/david/work_place/projects/auto_reload'

# Gems used only for assets and not required
# in production environments by default.
group :assets do
  gem 'sprockets-rails', github: 'rails/sprockets-rails'
  gem 'sass-rails',   github: 'rails/sass-rails'
  gem 'coffee-rails', github: 'rails/coffee-rails'
  gem 'uglifier', '>= 1.0.3'
end

gem 'jquery-rails', github: 'rails/jquery-rails'

# Deploy with Capistrano
# gem 'capistrano', group: :development

# To use debugger
gem 'debugger', group: [:development, :test]
\end{lstlisting}

可以看到,Auto-Reload技术的使用十分简单和直观。在代码\ref{auto-reload-demo}的12行中,通过一个gem语句指明了开发者想要使用Auto-Reload技术的意图,从而使得Auto—Reload作为一个Gem在服务器启动阶段被Rails载入到内存中。这之后,开发者需要在网页中手动加入JAvaScript代码订阅SSE消息并在接收到文件变动消息时指导浏览器刷新。

\subsection{Backend Instrumentation需求分析}

Backend Instrumentation则是另外一项开发者支持技术。它是用于实现Rails框架的网页式后台管理终端的一个主要技术之一。Backend Instrumentation支持实时的对Rails服务器进行性能评估和测试,并对评估和测试结果进行实时的计算和统计。通过收集和统计这些数据,能够使得开发者更为清晰地了解到Rails服务器后台目前的运行状态。这对于传统的后台开发来说并不是新的技术,对于传统的网页开发来说,却很难实现一个基于网页的后台管理和调试中断系统。Backend Instrumentation技术则在后台收集到数据之后,通过Rails消息总线技术,将数据传送到前端浏览器的消息订阅者处,从而实现了实时性能数据的推送,为Rails服务器前台网页式管理终端的实现提供了基础。

启用Backend Instrumentation技术依旧十分简单。首先需要确保本地机器上安装好了名为backendinstrument的Gem,然后开发者需要在Rails项目的Gemfile中指定使用该Gem,如下代码所示:

\begin{lstlisting}[caption={启用Auto-Realod技术}, label=backend-demo]
source 'https://rubygems.org'

gem 'rack', path: '/Users/david/work_place/projects/rack'
gem 'rails',     path: '/Users/david/work_place/projects/rails'
gem 'arel',      github: 'rails/arel'
gem 'activerecord-deprecated_finders', github: 'rails/activerecord-deprecated_finders'
gem 'debugger'
gem 'sqlite3'
gem 'puma'

# Enable Backend Instrumentation feature to get realtime performance data
gem 'backendinstrument', path: '/Users/david/work_place/projects/backend_instrument', group: [:deployment]

\end{lstlisting}

代码\ref{backend-demo}的12行指明了如何启用Backend Instrumentation技术。通过在Gemfile里使用gem语句,开发者告知Rails框架在启动时载入backendinstrument Gem,从而使得Backend Instrumentation技术的代码被载入至内存。这样,在程序运行期间,Backend Instrumentation技术将持续收集Rials服务器的性能数据,并将其推送至所有的前台消息订阅者处。值得指出的是,这里使用一个group子句,指明了Backend Instrumentation技术仅仅在网站部署时有效。这样设计是符合一般常识的,因为开发者一般仅希望在网站正式运行时才通过Backend Instrumentation技术抓去后台运行性能数据,从而实现远程监控和管理。

\section{解决方案概要}
基于上诉需求,详细描述本技术如何实现。特别是如何和Rails集成 指出本技术不仅仅依赖于对Rails库的改进,还需要笔者对其他开源库的改 进。并基于3.1的需求详细归纳出⼯工作点

\section{系统逻辑结构设计}
根据3.2的解决⽅方案归纳出各种实体类(系统静态结构)描述类之间的交互(系统动态结构)

\section{系统物理结构设计}
本系统开发部署环境(主要是软件环境)本系统得部署图(哪⼀一部分运⾏行在哪⾥里?)